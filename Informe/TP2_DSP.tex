
\documentclass[]{article}
%\documentclass[journal,10pt,draftclsnofoot,onecolumn]{IEEEtran}
%\usepackage{graphics,multirow,amsmath,amssymb,textcomp,subfigure,multirow,xspace,arydshln,cite}

\usepackage[]{graphicx}   % para manejar graficos

\usepackage[space]{grffile} % para manejar graficos

\usepackage{caption}

\usepackage{enumerate}    % para hacer listas numeradas

\usepackage{amsmath}        % no se..

\usepackage{amsfonts}     % no se..

\usepackage{authblk}    % para definir las afiliaciones de cada autor

\usepackage{layout}     % no se..

\usepackage[sorting=none]{biblatex}   % para manejar la bibliografia / referencias

\usepackage{lipsum}     % para generar texto random

\usepackage{multicol}   % para usar dos columnas

\usepackage{palatino}   % para que la fuente sea palatino

\usepackage[utf8]{inputenc} % para poder usar tildes

\usepackage[spanish]{babel} % para escribir en español

\addto\captionsspanish{\def\tablename{Tabla}} % cambiar ``cuadro'' por ``Tabla''

\usepackage[sc,big,raggedright,bf]{titlesec} % para definir el formato del header de cada seccion.

\usepackage[font=small]{caption} % para que la fuente de un epigrafe no tenga el mismo tamaño que el cuerpo del texto

\usepackage{geometry}
 \geometry{
 a4paper,
 textwidth={17cm},
 textheight={23cm},
 left={2cm},
 top={2.5cm},
 }

\setlength{\columnsep}{1cm} % para que la separacion entre columnas sea de 1 cm

\graphicspath {{imagenes/}}

\defbibheading{bibliography}{\section{\refname}} % para que bibtex no imponga su header cuando uso \printbibliography, y que se use el de babel

\addbibresource{bibliografia.bib} % para importar el archivo .bib

\title{\textbf{\LARGE{\textsf{DESCRIPTORES PARA ANÁLISIS AUTOMÁTICO DE MÚSICA}}}}
 % defino el titulo del Paper

\date{} % lo pongo vacio para que no aparezca abajo del abstract

\usepackage{fancyhdr}

\usepackage{hyperref}
%%%%%%%%%%%%%%%%%%%%%%%%%%%%%%%%%%%%%%%%%%%%%%%%%%%%%%%%%%%%%%%%%%%%%%%%%%%%%%%%%
% http://www-h.eng.cam.ac.uk/help/tpl/textprocessing/multicol_hint.html
\makeatletter           % esto lo uso para poder definir figuras
\newenvironment{tablehere}    % esto lo uso para poder definir figuras
  {\def\@captype{table}}    % esto lo uso para poder definir figuras

  {}              % esto lo uso para poder definir figuras
                  % esto lo uso para poder definir figuras
\newenvironment{figurehere}   % esto lo uso para poder definir figuras
  {\def\@captype{figure}}   % esto lo uso para poder definir figuras
  {\par\medskip}
  {}              % esto lo uso para poder definir figuras
\makeatother          % esto lo uso para poder definir figuras
%%%%%%%%%%%%%%%%%%%%%%%%%%%%%%%%%%%%%%%%%%%%%%%%%%%%%%%%%%%%%%%%%%%%%%%%%%%%%%%%%

%%%%%%%%%%%%%%%%%%%%%%%%%%%%%%%%%%%%%%%%%%%%%%%%%%%%%%%%%%%%%%%%%%%%%%%%%%%%%%%%%
%               ACA EMPIEZA EL DOCUMENTO                            %
%%%%%%%%%%%%%%%%%%%%%%%%%%%%%%%%%%%%%%%%%%%%%%%%%%%%%%%%%%%%%%%%%%%%%%%%%%%%%%%%%


\begin{document} % empieza el documentoo


\renewcommand{\headrulewidth}{0pt} % para que no haya linea decorativa en el header.


\author[1]{Federico Feldsberg} % defino el autor
\affil[1]{Universidad Nacional de Tres De Febrero, Buenos Aires, Argentina \newline \texttt{fedefelds@hotmail.com}} % afiliacion del autor


\begin{minipage}[h]{\textwidth} % uso el entorno minipage para que el abstract este en la misma pagina que el titulo
    \maketitle
    \thispagestyle{fancy}
    \fancyhf{}
    \rhead{19 de junio de 2017}
    \lhead{Procesamiento digital de señales}
    \cfoot{\thepage}

\end{minipage}


\begin{abstract}

\textit{Se hace un codigo que basicamente pude hacer muchas cosas. Se eligen 3 descriptores..\lipsum[1]}

\end{abstract}

\vspace{1.5cm}% Additional space between abstract & rest of document

\begin{multicols}{2}
\section{Introducción}
En este informe se describe el diseño y la implementacion de un sistema capaz de
analizar canciones y extraer informacion util de las mismas. Para ello se desarrllo
una serie de herramientas basadas en la libreria \textit{Librosa}. Dicha libreria
es de código abierto y esta validada por...AGREGAR VALIDACION...
\lipsum[2]
\lipsum[3]
Los objetivos de este trabajo son los siguientes:
\begin{itemize}
  \item Implementar un sistema que pueda remover el silencio al principio y al final de una señal
  \item Implementar un sistema que pueda normalizar la amplitud de una señal
  \item Implementar un sistema que pueda visualizar la STFT de una señal
  \item Implementar un sistema que pueda estimar el tempo de una señal
  \item Implementar un sistema que pueda calcular 3 descriptores a elegir
  \item Implementar un sistema que pueda normalizar los valores obtenidos
  \item Procesar un disco de música con las herramientas desarrolladas
\end{itemize}

\section{Descriptores Elegidos}
\subsection{Chroma Analisys}
El Chroma analisys fue introducido por primera vez por Fujishima en \cite{fujishima1999realtime}.
Dicho analisis es una manera de representar las caracteristicas espectrales
de una señal sonora. En dicha representacion, el espectro de frecuencias es
proyectado en 12 bins. Cada bin representa uno de los 12 distintos semitonos
de una octava musical. En otras palabras, todas las octavas de una nota musical
son mapeadas a uno de los 12 bins. Debido a esto es posible sintetizar, con cierta
pérdida de informacion, una señal a partir de su Chroma Analisys, mediante Chroma
Synthesis.

En \cite{Ellis} Ellis sostiene que el Chroma analisys puede dar informacion
util acerca de la señal en cuestion que no es evidente en el espectro original de
la señal. Por ejemplo, es capaz de señalar la similaridad musical percibida en
un tono de Shepard \cite{bello}.

Dicho analisis es implementado mediante el uso de la funcion \emph{{librosa.feature.chroma\_stft}}.
\subsection{Spectral Contrast}
Dada una señal musical, se presenta el problema de como identificar a que tipo de
musica pertenece la señal en cuestion. Este descriptor resulta util a la hora de
realizar esta tarea. Jiang indica que es posible que este descriptor puede llegar
a tener una mejor capacidad de discriminacion de tipos musicales que los MFCC
\cite{DanNingJiang}.

Octave-based Spectral Contrast considers the spectral peak, spectral valley and
their difference in each sub-band. For most music, the strong spectral peaks
roughly correspond with harmonic components; while non-harmonic components, or
noises, often appear at spectral valleys. Thus, Spectral Contrast feature
could roughly reflect the relative distribution of the harmonic and non-harmonic
components in the spectrum. Previous features, such as MFCC, average the
spectral distribution in each sub-band, and thus lose the relative spectral
information. Considering two spectra which have different spectral distribution
 may have similar average spectral characteristics, the average spectral
 distribution is not sufficient to represent the spectral characteristics of
 music. However, Spectral Contrast keeps more information and may have a better
 discrimination in music type classification.
\subsection{Descriptor 3}

\section{Arreglo experimental}
\subsection{Método 1}
\section{Resultados}





\section{Análisis de resultados}

\section{Conclusión}


\printbibliography
\end{multicols}



\end{document}
