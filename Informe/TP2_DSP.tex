
\documentclass[]{article}
%\documentclass[journal,10pt,draftclsnofoot,onecolumn]{IEEEtran}
%\usepackage{graphics,multirow,amsmath,amssymb,textcomp,subfigure,multirow,xspace,arydshln,cite}

\usepackage[]{graphicx}   % para manejar graficos

\usepackage[space]{grffile} % para manejar graficos

\usepackage{caption}

\usepackage{enumerate}    % para hacer listas numeradas

\usepackage{amsmath}        % no se..

\usepackage{amsfonts}     % no se..

\usepackage{authblk}    % para definir las afiliaciones de cada autor

\usepackage{layout}     % no se..

\usepackage{lineno}

\usepackage[sorting=none]{biblatex}   % para manejar la bibliografia / referencias

\usepackage{lipsum}     % para generar texto random

\usepackage{multicol}   % para usar dos columnas

\usepackage{palatino}   % para que la fuente sea palatino

\usepackage[utf8]{inputenc} % para poder usar tildes

\usepackage[spanish]{babel} % para escribir en español

\addto\captionsspanish{\def\tablename{Tabla}} % cambiar ``cuadro'' por ``Tabla''

\usepackage[sc,big,raggedright,bf]{titlesec} % para definir el formato del header de cada seccion.

\usepackage[font=small]{caption} % para que la fuente de un epigrafe no tenga el mismo tamaño que el cuerpo del texto

\usepackage{geometry}
 \geometry{
 a4paper,
 textwidth={17cm},
 textheight={23cm},
 left={2cm},
 top={2.5cm},
 }

\setlength{\columnsep}{1cm} % para que la separacion entre columnas sea de 1 cm

\graphicspath {{imagenes/}}

\defbibheading{bibliography}{\section{\refname}} % para que bibtex no imponga su header cuando uso \printbibliography, y que se use el de babel

\addbibresource{bibliografia.bib} % para importar el archivo .bib

\title{\textbf{\LARGE{\textsf{DESCRIPTORES PARA ANÁLISIS AUTOMÁTICO DE MÚSICA}}}}
 % defino el titulo del Paper

\date{} % lo pongo vacio para que no aparezca abajo del abstract

\newcommand{\figura}[3]{
\begin{figurehere}
\centering
\includegraphics[width=\linewidth]{#1}
\captionof{figure}{#2}
\label{#3}
\end{figurehere}
}
\newcommand{\tabla}[4]{
\begin{tablehere}
\begin{center}
\begin{tabular}{#1}
#2
\end{tabular}
\caption{#3}
\label{#4}
\end{center}
\end{tablehere}
}

\usepackage{fancyhdr}

\usepackage{hyperref}
%%%%%%%%%%%%%%%%%%%%%%%%%%%%%%%%%%%%%%%%%%%%%%%%%%%%%%%%%%%%%%%%%%%%%%%%%%%%%%%%%
% http://www-h.eng.cam.ac.uk/help/tpl/textprocessing/multicol_hint.html
\makeatletter           % esto lo uso para poder definir figuras
\newenvironment{tablehere}    % esto lo uso para poder definir figuras
  {\def\@captype{table}}    % esto lo uso para poder definir figuras

  {}              % esto lo uso para poder definir figuras
                  % esto lo uso para poder definir figuras
\newenvironment{figurehere}   % esto lo uso para poder definir figuras
  {\def\@captype{figure}}   % esto lo uso para poder definir figuras
  {\par\medskip}
  {}              % esto lo uso para poder definir figuras
\makeatother          % esto lo uso para poder definir figuras
%%%%%%%%%%%%%%%%%%%%%%%%%%%%%%%%%%%%%%%%%%%%%%%%%%%%%%%%%%%%%%%%%%%%%%%%%%%%%%%%%

%%%%%%%%%%%%%%%%%%%%%%%%%%%%%%%%%%%%%%%%%%%%%%%%%%%%%%%%%%%%%%%%%%%%%%%%%%%%%%%%%
%               ACA EMPIEZA EL DOCUMENTO                            %
%%%%%%%%%%%%%%%%%%%%%%%%%%%%%%%%%%%%%%%%%%%%%%%%%%%%%%%%%%%%%%%%%%%%%%%%%%%%%%%%%


\begin{document} % empieza el documentoo


\renewcommand{\headrulewidth}{0pt} % para que no haya linea decorativa en el header.


\author[1]{Federico Feldsberg} % defino el autor
\affil[1]{Universidad Nacional de Tres De Febrero, Buenos Aires, Argentina \newline \texttt{fedefelds@hotmail.com}} % afiliacion del autor


\begin{minipage}[h]{\textwidth} % uso el entorno minipage para que el abstract este en la misma pagina que el titulo
    \maketitle
    \thispagestyle{fancy}
    \fancyhf{}
    \rhead{19 de junio de 2017}
    \lhead{Procesamiento digital de señales}
    \cfoot{\thepage}

\end{minipage}


\begin{abstract}

\textit{Se hace un codigo que basicamente pude hacer muchas cosas. Se eligen 3 descriptores..\lipsum[1]}

\end{abstract}

\vspace{1.5cm}% Additional space between abstract & rest of document

\begin{multicols}{2}
\section{Introducción}
\label{sec: intro}
En este informe se describe el diseño y la implementacion de un sistema capaz de
analizar canciones y extraer informacion util de las mismas. Para ello se desarrllo
una serie de herramientas basadas en la libreria \textit{Librosa}. Dicha libreria
es de código abierto y esta validada por...AGREGAR VALIDACION...
\lipsum[2]
\lipsum[3]
Los objetivos de este trabajo son los siguientes:
\begin{itemize}
  \item Implementar un sistema que pueda remover el silencio al principio y al final de una señal
  \item Implementar un sistema que pueda normalizar la amplitud de una señal
  \item Implementar un sistema que pueda visualizar la STFT de una señal
  \item Implementar un sistema que pueda estimar el tempo de una señal
  \item Implementar un sistema que pueda calcular 3 descriptores a elegir
  \item Procesar un disco de música con las herramientas desarrolladas
  \item Implementar un sistema que pueda normalizar los valores obtenidos
\end{itemize}

\section{Descriptores Elegidos}

\subsection{Tempo}
La estimacion del tempo es fundamental para el procesamiento automatico de musica.
Segun Alonso et al. \cite{alonso2004tempo}, el tempo es un elemento que sustenta
la musica occidental, y por lo tanto su comprension y modelado es de gran interes
en el campo del procesamiento automatico de musica.

Es por eso que hoy en dia existen varios métodos disponibles para estimar el tempo
de una cancion \cite{goto1997issues}. En este caso, se implementa este descriptor
mediante el uso de la funcion \emph{{librosa.feature.chroma\_stft}}.
\subsection{Factor de cresta}
Segun \cite{ wiki:crest}{}, el factor de cresta es una manera de medir que tan
pronunciados son los picos de una señal. Un factor de cresta igual a 1 indica
la ausencia de picos en la señal.

Dada una señal, el factor de cresta de la misma esta dado por
\begin{equation*}
  FC=\frac{\left|S_{max}\right|}{S_{rms}}
\end{equation*}

Donde $S_{max}$ es maximo valor que toma la señal y $S_{rms}$ es su valor medio
cuadratico.


\subsection{Silence Rate}
Dicho descriptor es experimental. Constiuye una manera de medir que tan
silenciosa es una canción. Sea una señal de $N$ muestras. Se propone dividir la
canción en intervalos silenciosos y no silenciosos. Se obtiene que de $N$ muestras,
$M_s$ son silenciosas y $N-M_s$ no lo son.

El Silence Rate esta dado por la relacion entre la cantidad de muestras
silenciosas y la cantidad total de muestras:

\begin{equation}
  SR=\frac{N-M_s}{N}
\end{equation}

Como  0 $\leq N-M_s\leq N$, entonces $0 \leq SR \leq 1$.
Una canción cuyo $SR$ es 1 corresponde a una señal totalmente silenciosa.
Similarmente, una cancion cuyo $SR$ es nulo corresponde a una señal totalmente
no silenciosa.

En este trabajo, dicho analisis es implementado mediante el uso del metodo
\emph{{librosa.effects.split}}.

\subsection{Chroma Analisys}
\label{Sec:Chroma}
El Chroma analisys fue introducido por primera vez por Fujishima en \cite{fujishima1999realtime}.
Dicho analisis es una manera de representar las caracteristicas espectrales
de una señal sonora. En dicha representacion, el espectro de frecuencias es
proyectado en 12 bins. Cada bin representa uno de los 12 distintos semitonos
de una octava musical. En otras palabras, todas las octavas de una nota musical
son mapeadas a uno de los 12 bins. Debido a esto es posible sintetizar, con cierta
pérdida de informacion, una señal a partir de su Chroma Analisys, mediante Chroma
Synthesis.

En \cite{Ellis} Ellis sostiene que el Chroma analisys puede dar informacion
util acerca de la señal en cuestion que no es evidente en el espectro original de
la señal. Por ejemplo, es capaz de señalar la similaridad musical percibida en
un tono de Shepard \cite{bello}.

En este trabajo, dicho analisis es implementado mediante el uso de la funcion \emph{{librosa.feature.chroma\_stft}}.
% \subsection{Spectral Contrast}
% Dada una señal musical, se presenta el problema de como identificar a que tipo de
% musica pertenece la señal en cuestion. Este descriptor resulta util a la hora de
% realizar esta tarea. Jiang et al. indica que es posible que este descriptor puede llegar
% a tener una mejor capacidad de discriminacion de tipos musicales que los MFCC
% \cite{DanNingJiang}.
%
% Dicho descriptor considera las diferencias entre los picos espectrales y los valles
% espectrales para cada sub banda. En la mayoria de los casos en los que se analiza
% musica, los fuertes picos corresponden aproximadamente con las componentes armonicas
% mientras que las componentes no armonicas o ruidos corresponden con los valles.
% Por eso el descriptor en cuestión permite caracterizar la distribucion relativa
% de las componentes armonicas y no armonicas del espectro.
%
% En este trabajo, dicho analisis es implementado mediante el uso de la funcion \emph{{librosa.feature.spectral\_contrast}}.
% \subsection{Tempogram}
%
%
% Este descriptor se basa en un tratamiento similar al presentado en la seccion
% \ref{Sec:Chroma}. En el caso de dicha seccion, se recurre a mapear varias octavas
% o harmonicos a una determinada cantidad de bins \cite{kurth2006cyclic}. En el caso del tempograma se
% recurre a mapear varios tempos que difieren por un factor de 2 en un mismo bin.
% En terminos analogos, podemos pensar en los armonicos de una frecuencia dada,
% que no son mas que frecuencias relacionadas por un factor de 2.
%
% En este trabajo, dicho analisis es implementado mediante el uso del metodo
% \emph{{librosa.feature.tempogram}}.

\section{Resultados}
En la siguiente sección se describe la implementacion de los objetivos propuestos
en la sección \ref{sec: intro}. El álbum a analizar es \emph{The Turn of a Friendly
card  \emph{de} The Alan Parsons Project}. Dicho álbum consta de las siguientes
canciones:
\begin{enumerate}
  \item May Be a Price to Pay
  \item Games People Play
  \item Time
  \item I Don't Wanna Go Home
  \item The Gold Bug
  \item The Turn of a Friendly Card: The Turn of a Friendly Card (Part 1)
  \item The Turn of a Friendly Card: Snake Eyes
  \item The Turn of a Friendly Card: The Ace Of Swords
  \item The Turn of a Friendly Card: Nothing Left To Lose
  \item The Turn of a Friendly Card: The Turn of a Friendly Card (Part 2)
  

\end{enumerate}

\subsection{Preparación de la señal temporal}
%  Implemente un sistema que pueda cargar sen ̃ales de audio, remover el
%  silencio al principio y al final de las mismas y normalizarlas.
%  Esta tarea puede verse diagramada en los bloques violetas de la Figura 1.
\label{sec: prep}
La preparacion de la señal consistie en cargar el archivo, extraer el silencio
al principio y al final y finalmente, normalizar la señal. Dichas tareas se
implementan mediante un script en python disponible en \


\subsection{Visualización de la STFT}
\label{sec: stft}
Para la implementacion de la STFT, se considera el siguiente script, propuesto
en \cite{stft}:


El resultado obtenido es la siguiente figura:
\figura
{stft}
{Visualización de la STFT}
{fig: stft}

\subsection{Estimación del tempo}
\label{sec: tempo}

La estimacion del tempo es fundamental para el procesamiento automatico de musica.
Segun Alonso et al. \cite{alonso2004tempo}, el tempo es un elemento que sustenta
la musica occidental, y por lo tanto su comprension y modelado es de gran interes
en el campo del procesamiento automatico de musica.

Es por eso que hoy en dia existen varias opciones a la hora de estimar el tempo
de una cancion \cite{goto1997issues}. En este caso, se implementa este descriptor
mediante el uso de la funcion \emph{{librosa.feature.chroma\_stft}}.

Los resultados obtenidos son:

La implementacion de dicho descriptor es detallada en el anexo \ref{tempo}

\subsection{Factor de cresta}
Los resultados obtenidos son:

La implementacion de dicho descriptor es detallada en el anexo \ref{tempo}

\subsection{Silence Rate}

\subsection{Chroma Analysis}


% \section{Análisis de resultados}


\section{Conclusión}
\printbibliography
\end{multicols}

\newpage

\appendix
\section{Resultados adicionales}

\tabla
{|l|c|c|c|c|c|c|c|c|c|c|}
{
\hline
 &Tema 1 & Tema 2 & Tema 3 & Tema 4 & Tema 5 & Tema 6 & Tema 7 & Tema 8 & Tema 9 & Tema 10 \\
\hline
Bin 1 & 0.504 & 0.530 & 0.457 & 0.570 & 0.417 & 0.324 & 0.551 & 0.459 & 0.406 & 0.351 \\
\hline
Bin 2 & 0.507 & 0.449 & 0.436 & 0.470 & 0.514 & 0.463 & 0.505 & 0.489 & 0.392 & 0.489 \\
\hline
Bin 3 & 0.559 & 0.484 & 0.562 & 0.423 & 0.683 & 0.609 & 0.507 & 0.576 & 0.464 & 0.645 \\
\hline
Bin 4 & 0.523 & 0.428 & 0.712 & 0.463 & 0.486 & 0.472 & 0.558 & 0.483 & 0.470 & 0.494 \\
\hline
Bin 5 & 0.618 & 0.431 & 0.496 & 0.543 & 0.434 & 0.451 & 0.650 & 0.501 & 0.487 & 0.514 \\
\hline
Bin 6 & 0.704 & 0.458 & 0.365 & 0.428 & 0.489 & 0.402 & 0.525 & 0.514 & 0.512 & 0.490 \\
\hline
Bin 7 & 0.521 & 0.602 & 0.381 & 0.430 & 0.478 & 0.343 & 0.475 & 0.512 & 0.495 & 0.384 \\
\hline
Bin 8 & 0.458 & 0.579 & 0.447 & 0.556 & 0.531 & 0.371 & 0.502 & 0.555 & 0.462 & 0.418 \\
\hline
Bin 9 & 0.494 & 0.469 & 0.399 & 0.564 & 0.550 & 0.471 & 0.602 & 0.512 & 0.396 & 0.532 \\
\hline
Bin 10 & 0.570 & 0.476 & 0.428 & 0.627 & 0.626 & 0.638 & 0.765 & 0.579 & 0.550 & 0.723 \\
\hline
Bin 11 & 0.466 & 0.583 & 0.543 & 0.478 & 0.455 & 0.457 & 0.580 & 0.518 & 0.704 & 0.509 \\
\hline
Bin 12 & 0.427 & 0.726 & 0.464 & 0.500 & 0.376 & 0.308 & 0.511 & 0.446 & 0.570 & 0.328 \\
\hline
}
{Valores del descriptor chroma obtenidos sin normalización}
{tab: chroma}

\tabla
{|l|c|c|c|c|c|c|c|c|c|c|}
{
\hline
&Tema 1 & Tema 2 & Tema 3 & Tema 4 & Tema 5 & Tema 6 & Tema 7 & Tema 8 & Tema 9 & Tema 10 \\
\hline
Bin 1  & 0.659 & 0.693 & 0.597 & 0.745 & 0.545 & 0.424 & 0.720 & 0.600 & 0.531 & 0.459 \\
\hline
Bin 2  & 0.663 & 0.587 & 0.570 & 0.614 & 0.672 & 0.605 & 0.660 & 0.639 & 0.512 & 0.639 \\
\hline
Bin 3  & 0.731 & 0.633 & 0.735 & 0.553 & 0.893 & 0.796 & 0.663 & 0.753 & 0.607 & 0.843 \\
\hline
Bin 4  & 0.684 & 0.559 & 0.931 & 0.605 & 0.635 & 0.617 & 0.729 & 0.631 & 0.614 & 0.646 \\
\hline
Bin 5  & 0.808 & 0.563 & 0.648 & 0.710 & 0.567 & 0.590 & 0.850 & 0.655 & 0.637 & 0.672 \\
\hline
Bin 6  & 0.920 & 0.599 & 0.477 & 0.559 & 0.639 & 0.525 & 0.686 & 0.672 & 0.669 & 0.641 \\
\hline
Bin 7  & 0.681 & 0.787 & 0.498 & 0.562 & 0.625 & 0.448 & 0.621 & 0.669 & 0.647 & 0.502 \\
\hline
Bin 8  & 0.599 & 0.757 & 0.584 & 0.727 & 0.694 & 0.485 & 0.656 & 0.725 & 0.604 & 0.546 \\
\hline
Bin 9  & 0.646 & 0.613 & 0.522 & 0.737 & 0.719 & 0.616 & 0.787 & 0.669 & 0.518 & 0.695 \\
\hline
Bin 10 & 0.745 & 0.622 & 0.559 & 0.820 & 0.818 & 0.834 & 1.000 & 0.757 & 0.719 & 0.945 \\
\hline
Bin 11 & 0.609 & 0.762 & 0.710 & 0.625 & 0.595 & 0.597 & 0.758 & 0.677 & 0.920 & 0.665 \\
\hline
Bin 12 & 0.558 & 0.949 & 0.607 & 0.654 & 0.492 & 0.403 & 0.668 & 0.583 & 0.745 & 0.429 \\
\hline
}
{Valores del descriptor chroma obtenidos normalizados}
{tab: chroma}

\section{Implementaciones}
\subsection{Preparación de la señal temporal}
\label{Prep}
\begin{verbatim}
import librosa
import numpy as np
import matplotlib.pyplot as plt
carpeta='/Users/Fede/Desktop/The Turn of a Friendly Card 1979 (GPF)/Canciones del trabajo/'
filename='10'
formato='.mp3'
filename=carpeta+filename+formato
# cargar audio
y, sr = librosa.load(filename)
# extraer silencios al principio y final
y,index=librosa.effects.trim(y, top_db=60, ref=np.amax, frame_length=2048, hop_length=50)
# normalizar
valor_max=np.max(y)
y=y/valor_max
\end{verbatim}
\subsection{Visualización de la STFT}
\label{stft}
\begin{verbatim}
import librosa
import numpy as np
import matplotlib.pyplot as plt
carpeta='/Users/Fede/Desktop/'
filename='burno mars'
formato='.mp3'
filename=carpeta+filename+formato
# cargar audio
y, sr = librosa.load(filename)
# extraer silencios al principio y final
y,index=librosa.effects.trim(y, top_db=60, ref=np.amax, frame_length=2048, hop_length=50)
# normalizar
valor_max=np.max(y)
y=y/valor_max
D = librosa.stft(y)
librosa.display.specshow(librosa.amplitude_to_db(D,ref=np.max)
,y_axis='log', x_axis='time')
plt.title('Power spectrogram')
plt.colorbar(format='%+2.0f dB')
plt.tight_layout()
plt.show()
\end{verbatim}

\subsection{Descriptor Tempo}
\label{tempo}
\begin{verbatim}
import librosa
import numpy as np

carpeta='/Users/Fede/Desktop/The Turn of a Friendly Card 1979 (GPF)/Canciones del trabajo/'
filename='10'
formato='.mp3'
filename=carpeta+filename+formato
# cargar audio
y, sr = librosa.load(filename)
# extraer silencios al principio y final
y,index=librosa.effects.trim(y, top_db=60, ref=np.amax, frame_length=2048, hop_length=50)
# normalizar
valor_max=np.max(y)
y=y/valor_max
# calcular tempo
hop_length = 512
oenv = librosa.onset.onset_strength(y=y, sr=sr, hop_length=hop_length)
tempogram = librosa.feature.tempogram(onset_envelope=oenv, sr=sr,
                                      hop_length=hop_length)
ac_global = librosa.autocorrelate(oenv, max_size=tempogram.shape[0])
ac_global = librosa.util.normalize(ac_global)
# Estimate the global tempo for display purposes
tempo = librosa.beat.tempo(onset_envelope=oenv, sr=sr,hop_length=hop_length)[0]
print(tempo)
\end{verbatim}
\subsection{Silence Rate}
\label{silence rate}
\begin{verbatim}
import librosa
import numpy as np
carpeta='/Users/Fede/Desktop/The Turn of a Friendly Card 1979 (GPF)/Canciones del trabajo/'
filename='9'
formato='.mp3'
filename=carpeta+filename+formato
# cargar audio
y, sr = librosa.load(filename)
# # extraer silencios al principio y final
# y,index=librosa.effects.trim(y, top_db=60, ref=np.amax,
# frame_length=1024, hop_length=50)
# normalizar
valor_max=np.max(y)
y=y/valor_max
# calcular sr
intervals=librosa.effects.split(y, top_db=60, ref=np.amax,
frame_length=1024, hop_length=50)
M_ns=0
for i in range(0,intervals.shape[0]):
    M_ns = M_ns+intervals[i,1]-intervals[i,0]
N=len(y)
SR=((N-M_ns)/N)
SR=np.array([SR])
print(SR)
\end{verbatim}

\subsection{Factor de cresta}
\label{factor de cresta}
\begin{verbatim}
import librosa
import numpy as np
import matplotlib.pyplot as plt
carpeta='/Users/Fede/Desktop/The Turn of a Friendly Card 1979 (GPF)/Canciones del trabajo/'
filename='10'
formato='.mp3'
filename=carpeta+filename+formato
# cargar audio
y, sr = librosa.load(filename)
# extraer silencios al principio y final
y,index=librosa.effects.trim(y, top_db=60, ref=np.amax, frame_length=2048, hop_length=50)
# normalizar
valor_max=np.max(y)
y=y/valor_max
# calculo el valor rms
y_rms=y*y
y_rms=np.sum(y_rms)
y_rms=y_rms/len(y)
y_rms=np.sqrt(y_rms)
# calculo el valor maximo
y_max=np.max(y) # siempre vale 1
# calculo el factor de cresta
factor_de_cresta=((y_max)/(y_rms))
print(factor_de_cresta)
\end{verbatim}

\subsection{Chroma}
\label{chroma}
\begin{verbatim}
import librosa
import numpy as np
carpeta='/Users/Fede/Desktop/The Turn of a Friendly Card 1979 (GPF)/Canciones del trabajo/'
filename='10'
formato='.mp3'
filename=carpeta+filename+formato
# cargar audio
y, sr = librosa.load(filename)
# extraer silencios al principio y final
y,index=librosa.effects.trim(y, top_db=60, ref=np.amax,
frame_length=1024, hop_length=50)
# normalizar
valor_max=np.max(y)
y=y/valor_max
# calculo el chromagram
chromagram=librosa.feature.chroma_cqt(y,sr)
chromagram=np.mean(chromagram,1)
np.savetxt('test.txt',chromagram,delimiter=' \\ ',fmt='%.3f', newline=' & ')
\end{verbatim}

\end{document}
