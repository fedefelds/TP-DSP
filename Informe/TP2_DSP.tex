
\documentclass[]{article}
%\documentclass[journal,10pt,draftclsnofoot,onecolumn]{IEEEtran}
%\usepackage{graphics,multirow,amsmath,amssymb,textcomp,subfigure,multirow,xspace,arydshln,cite}

\usepackage[]{graphicx}   % para manejar graficos

\usepackage[space]{grffile} % para manejar graficos

\usepackage{caption}

\usepackage{enumerate}    % para hacer listas numeradas

\usepackage{amsmath}        % no se..

\usepackage{amsfonts}     % no se..

\usepackage{authblk}    % para definir las afiliaciones de cada autor

\usepackage{layout}     % no se..

\usepackage{lineno}

\usepackage[sorting=none]{biblatex}   % para manejar la bibliografia / referencias

\usepackage{lipsum}     % para generar texto random

\usepackage{multicol}   % para usar dos columnas

\usepackage{palatino}   % para que la fuente sea palatino

\usepackage[utf8]{inputenc} % para poder usar tildes

\usepackage[spanish]{babel} % para escribir en español

\addto\captionsspanish{\def\tablename{Tabla}} % cambiar ``cuadro'' por ``Tabla''

\usepackage[sc,big,raggedright,bf]{titlesec} % para definir el formato del header de cada seccion.

\usepackage[font=small]{caption} % para que la fuente de un epigrafe no tenga el mismo tamaño que el cuerpo del texto

\usepackage{geometry}
 \geometry{
 a4paper,
 textwidth={17cm},
 textheight={23cm},
 left={2cm},
 top={2.5cm},
 }

\setlength{\columnsep}{1cm} % para que la separacion entre columnas sea de 1 cm

\graphicspath {{imagenes/}}

\defbibheading{bibliography}{\section{\refname}} % para que bibtex no imponga su header cuando uso \printbibliography, y que se use el de babel

\addbibresource{bibliografia.bib} % para importar el archivo .bib

\title{\textbf{\LARGE{\textsf{DESCRIPTORES PARA ANÁLISIS AUTOMÁTICO DE MÚSICA}}}}
 % defino el titulo del Paper

\date{} % lo pongo vacio para que no aparezca abajo del abstract

\newcommand{\figura}[3]{
\begin{figurehere}
\centering
\includegraphics[width=\linewidth]{#1}
\captionof{figure}{#2}
\label{#3}
\end{figurehere}
}


\usepackage{fancyhdr}

\usepackage{hyperref}
%%%%%%%%%%%%%%%%%%%%%%%%%%%%%%%%%%%%%%%%%%%%%%%%%%%%%%%%%%%%%%%%%%%%%%%%%%%%%%%%%
% http://www-h.eng.cam.ac.uk/help/tpl/textprocessing/multicol_hint.html
\makeatletter           % esto lo uso para poder definir figuras
\newenvironment{tablehere}    % esto lo uso para poder definir figuras
  {\def\@captype{table}}    % esto lo uso para poder definir figuras

  {}              % esto lo uso para poder definir figuras
                  % esto lo uso para poder definir figuras
\newenvironment{figurehere}   % esto lo uso para poder definir figuras
  {\def\@captype{figure}}   % esto lo uso para poder definir figuras
  {\par\medskip}
  {}              % esto lo uso para poder definir figuras
\makeatother          % esto lo uso para poder definir figuras
%%%%%%%%%%%%%%%%%%%%%%%%%%%%%%%%%%%%%%%%%%%%%%%%%%%%%%%%%%%%%%%%%%%%%%%%%%%%%%%%%

%%%%%%%%%%%%%%%%%%%%%%%%%%%%%%%%%%%%%%%%%%%%%%%%%%%%%%%%%%%%%%%%%%%%%%%%%%%%%%%%%
%               ACA EMPIEZA EL DOCUMENTO                            %
%%%%%%%%%%%%%%%%%%%%%%%%%%%%%%%%%%%%%%%%%%%%%%%%%%%%%%%%%%%%%%%%%%%%%%%%%%%%%%%%%


\begin{document} % empieza el documentoo


\renewcommand{\headrulewidth}{0pt} % para que no haya linea decorativa en el header.


\author[1]{Federico Feldsberg} % defino el autor
\affil[1]{Universidad Nacional de Tres De Febrero, Buenos Aires, Argentina \newline \texttt{fedefelds@hotmail.com}} % afiliacion del autor


\begin{minipage}[h]{\textwidth} % uso el entorno minipage para que el abstract este en la misma pagina que el titulo
    \maketitle
    \thispagestyle{fancy}
    \fancyhf{}
    \rhead{19 de junio de 2017}
    \lhead{Procesamiento digital de señales}
    \cfoot{\thepage}

\end{minipage}


\begin{abstract}

\textit{Se hace un codigo que basicamente pude hacer muchas cosas. Se eligen 3 descriptores..\lipsum[1]}

\end{abstract}

\vspace{1.5cm}% Additional space between abstract & rest of document

\begin{multicols}{2}
\section{Introducción}
\label{sec: intro}
En este informe se describe el diseño y la implementacion de un sistema capaz de
analizar canciones y extraer informacion util de las mismas. Para ello se desarrllo
una serie de herramientas basadas en la libreria \textit{Librosa}. Dicha libreria
es de código abierto y esta validada por...AGREGAR VALIDACION...
\lipsum[2]
\lipsum[3]
Los objetivos de este trabajo son los siguientes:
\begin{itemize}
  \item Implementar un sistema que pueda remover el silencio al principio y al final de una señal
  \item Implementar un sistema que pueda normalizar la amplitud de una señal
  \item Implementar un sistema que pueda visualizar la STFT de una señal
  \item Implementar un sistema que pueda estimar el tempo de una señal
  \item Implementar un sistema que pueda calcular 3 descriptores a elegir
  \item Procesar un disco de música con las herramientas desarrolladas
  \item Implementar un sistema que pueda normalizar los valores obtenidos
\end{itemize}

\section{Descriptores Elegidos}

\subsection{Silence Rate}
Dicho descriptor es experimental. Constiuye una manera de medir que tan
silenciosa es una canción. Sea una señal de $N$ muestras. Se propone dividir la
canción en intervalos silenciosos y no silenciosos. Se obtiene que de $N$ muestras,
$M_s$ son silenciosas y $N-M_s$ no lo son.

El Silence Rate esta dado por la relacion entre la cantidad de muestras
silenciosas y la cantidad total de muestras:

\begin{equation}
  SR=\frac{N-M_s}{N}
\end{equation}

Como $N-M_s\leq N$, entonces $0 \leq SR \leq 1$.
Una canción cuyo $SR$ es 1 corresponde a una señal totalmente silenciosa.
Similarmente, una cancion cuyo $SR$ es nulo corresponde a una señal totalmente
no silenciosa.

En este trabajo, dicho analisis es implementado mediante el uso del metodo
\emph{{librosa.effects.split}}.

\subsection{Chroma Analisys}
\label{Sec:Chroma}
El Chroma analisys fue introducido por primera vez por Fujishima en \cite{fujishima1999realtime}.
Dicho analisis es una manera de representar las caracteristicas espectrales
de una señal sonora. En dicha representacion, el espectro de frecuencias es
proyectado en 12 bins. Cada bin representa uno de los 12 distintos semitonos
de una octava musical. En otras palabras, todas las octavas de una nota musical
son mapeadas a uno de los 12 bins. Debido a esto es posible sintetizar, con cierta
pérdida de informacion, una señal a partir de su Chroma Analisys, mediante Chroma
Synthesis.

En \cite{Ellis} Ellis sostiene que el Chroma analisys puede dar informacion
util acerca de la señal en cuestion que no es evidente en el espectro original de
la señal. Por ejemplo, es capaz de señalar la similaridad musical percibida en
un tono de Shepard \cite{bello}.

En este trabajo, dicho analisis es implementado mediante el uso de la funcion \emph{{librosa.feature.chroma\_stft}}.
\subsection{Spectral Contrast}
Dada una señal musical, se presenta el problema de como identificar a que tipo de
musica pertenece la señal en cuestion. Este descriptor resulta util a la hora de
realizar esta tarea. Jiang et al. indica que es posible que este descriptor puede llegar
a tener una mejor capacidad de discriminacion de tipos musicales que los MFCC
\cite{DanNingJiang}.

Dicho descriptor considera las diferencias entre los picos espectrales y los valles
espectrales para cada sub banda. En la mayoria de los casos en los que se analiza
musica, los fuertes picos corresponden aproximadamente con las componentes armonicas
mientras que las componentes no armonicas o ruidos corresponden con los valles.
Por eso el descriptor en cuestión permite caracterizar la distribucion relativa
de las componentes armonicas y no armonicas del espectro.

En este trabajo, dicho analisis es implementado mediante el uso de la funcion \emph{{librosa.feature.spectral\_contrast}}.
\subsection{Tempogram}


Este descriptor se basa en un tratamiento similar al presentado en la seccion
\ref{Sec:Chroma}. En el caso de dicha seccion, se recurre a mapear varias octavas
o harmonicos a una determinada cantidad de bins \cite{kurth2006cyclic}. En el caso del tempograma se
recurre a mapear varios tempos que difieren por un factor de 2 en un mismo bin.
En terminos analogos, podemos pensar en los armonicos de una frecuencia dada,
que no son mas que frecuencias relacionadas por un factor de 2.

En este trabajo, dicho analisis es implementado mediante el uso del metodo
\emph{{librosa.feature.tempogram}}.

\section{Resultados}
En la siguiente sección se describe la implementacion de los objetivos propuestos
en la sección \ref{sec: intro}. En las subsecciones \ref{sec: prep}, \ref{sec: stft}
 y \ref{sec: tempo} se utiliza la canción \emph{Brain Stew} de GreenDay como
señal de prueba.
\subsection{Preparación de la señal temporal}
\label{sec: prep}
La preparacion de la señal consistie en cargar el archivo, extraer el silencio
al principio y al final y finalmente, normalizar la señal. Dichas tareas se
implementan mediante un script en python:


\subsection{Visualización de la STFT}
\label{sec: stft}
Para la implementacion de la STFT, se considera el siguiente script, propuesto
en \cite{stft}:


El resultado obtenido es la siguiente figura:
\figura
{stft}
{Visualización de la STFT}
{fig: stft}
\subsection{Estimación del tempo}
\label{sec: tempo}
Para la estimacion del tempo se implementa el siguiente script:


\section{Análisis de resultados}


\section{Conclusión}
\printbibliography
\end{multicols}

\newpage
\appendix

\section{Implementaciones}
En las siguientes subsecciones se exponen los scripts utilizados a lo largo de
este trabajo.

\subsection{Preparación de la señal temporal}
\begin{verbatim}
import librosa
import numpy as np
filename='/Users/Fede/Desktop/stew.mp3'
# cargar audio
y, sr = librosa.load(filename)
# extraer silencios al principio y final
y,index=librosa.effects.trim(y, top_db=60, ref=np.amax, frame_length=2048, hop_length=50)
# normalizar
valor_max=np.max(y)
y=y/valor_max
\end{verbatim}

\subsection{Visualización de la STFT}
\begin{verbatim}
import librosa
import numpy as np
import matplotlib.pyplot as plt
filename='/Users/Fede/Desktop/stew.mp3'
# cargar audio
y, sr = librosa.load(filename)
# extraer silencios al principio y final
y,index=librosa.effects.trim(y, top_db=60, ref=np.amax, frame_length=2048, hop_length=50)
# normalizar
valor_max=np.max(y)
y=y/valor_max
D = librosa.stft(y)
librosa.display.specshow(librosa.amplitude_to_db(D,ref=np.max)
,y_axis='log', x_axis='time')
plt.title('Power spectrogram')
plt.colorbar(format='%+2.0f dB')
plt.tight_layout()
plt.show()
\end{verbatim}

\subsection{Estimación del tempo}
\begin{verbatim}
  import librosa
  import numpy as np
  filename='/Users/Fede/Desktop/stew.mp3'
  # cargar audio
  y, sr = librosa.load(filename)
  # extraer silencios al principio y final
  y,index=librosa.effects.trim(y, top_db=60, ref=np.amax, frame_length=2048, hop_length=50)
  # normalizar
  valor_max=np.max(y)
  y=y/valor_max
  # calcular tempo
  hop_length = 512
  oenv = librosa.onset.onset_strength(y=y, sr=sr, hop_length=hop_length)
  tempogram = librosa.feature.tempogram(onset_envelope=oenv, sr=sr,
                                        hop_length=hop_length)
  ac_global = librosa.autocorrelate(oenv, max_size=tempogram.shape[0])
  ac_global = librosa.util.normalize(ac_global)
  # Estimate the global tempo for display purposes
  tempo = librosa.beat.tempo(onset_envelope=oenv, sr=sr,hop_length=hop_length)[0]
  print(tempo)
\end{verbatim}





\end{document}
